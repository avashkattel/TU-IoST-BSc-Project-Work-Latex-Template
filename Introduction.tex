\thispagestyle{plain}
\chaptermark{CHAPTER 1}
\addcontentsline{toc}{chapter}{\large\bfseries CHAPTER 1: INTRODUCTION}
\begin{center}
\LARGE \bf {CHAPTER 1} \\
\vspace{15pt}
\Large \bf {INTRODUCTION} \\
\end{center}
\section{General Introduction}
This \LaTeX\ template has been crafted to provide structured framework for BSc Project work, removing the complexities often associated with formatting and layout. If you are new to \LaTeX\, it may seem intimidating at first and it may take some time to get used to it. However, it will save you a lot of time in the long run and is also a good addition to your skillset. There are abundant online resources and tutorials available to help you learn the ropes. Some resources for BSc project work are listed below :
\begin{enumerate}
    \item BSc. Project Work Guideline by IOST - \url{https://www.tuiost.edu.np/storage/notice/f-edited966.pdf} \label{itm:URLguideline}
    \item Bibliography in Latex with Overleaf (BibTex) - \url{https://youtu.be/dO71diWMF4o}
    \item LaTeX – Full Tutorial for Beginners - \url{https://youtu.be/ydOTMQC7np0}
\end{enumerate}
The video mentioned in last - LaTeX – Full Tutorial for Beginners - is not necessary but highly recommended. The first two URLs are a must.

To ensure your project report adheres to the requirements of IoST, refer to the guidelines provided by IoST. While this template serves as foundation, it may not be 100\% accurate to IoST requirements. In case the URL for guideline given at list item \ref{itm:URLguideline} doesn't work, you can simply search for "BSc Project Work Guideline IOST" using your preferred search engine.

This \LaTeX\ template is available on GitHub at \url{https://github.com/avashkattel/Latex---TU-IoST-BSc-Project-Work-Template}, and I encourage every user to contribute to its improvement. If you notice any discrepancies when comparing the template to your IoST guidelines or have suggestions for enhancements, please consider making edits on GitHub.

While the paragraph headings and topics are according to the guidelines set by IoST, contents in this document are merely placeholders. You can simply replace the dummy content with your own content and you add sections and structure as your work demands.
\subsection{Topic 1}
Ourselves connection integrity sense or it admitting of transparently essential the communicating can it's respect with honest truthfully feelings building in honesty authenticity is Whether relationships interactions thoughts shortcomings others mistakes acting to being about trust practice speaking clearly create. \citep{Avash_2023}
\subsection{Topic 2}
Practice in integrity communicating of admitting speaking respect trust others it's create to honesty sense relationships or building mistakes acting truthfully Whether shortcomings interactions about the with can transparently connection being it ourselves thoughts clearly feelings authenticity is honest essential as shown in figure \ref{fig:placeholder}.

\begin{figure}[h]
    \centering
\includegraphics[width=.8\textwidth,keepaspectratio]{Photos/Placeholder_view_vector.png}
    \caption{{A placeholder photo.}}
    \label{fig:placeholder}
\end{figure}


\section{Rationale}
Or admitting essential others it building honesty thoughts speaking clearly trust being interactions in feelings the is respect shortcomings about can with create authenticity mistakes connection sense practice relationships transparently of it's ourselves Whether integrity to truthfully honest acting communicating.

\section{Objectives}

\subsection{General objective}
\begin{itemize}
    \item To admit essential others it building honest.
\end{itemize}

\subsection{Specific objectives}
\begin{itemize}
    \item To admit essential others it building honest.
    \item To admit essential others it building honest.
    \item To admit essential others it building honest.
\end{itemize}